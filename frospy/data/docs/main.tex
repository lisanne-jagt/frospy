\documentclass{article}
\usepackage[utf8]{inputenc}
\usepackage{listings}
\usepackage{color}

%Everything to write code
\definecolor{dkgreen}{rgb}{0,0.6,0}
\definecolor{gray}{rgb}{0.5,0.5,0.5}
\definecolor{mauve}{rgb}{0.58,0,0.82}

\lstset{frame=tb,
  language=python,
  aboveskip=3mm,
  belowskip=3mm,
  showstringspaces=false,
  columns=flexible,
  basicstyle={\small\ttfamily},
  numbers=none,
  numberstyle=\tiny\color{gray},
  keywordstyle=\color{blue},
  commentstyle=\color{dkgreen},
  stringstyle=\color{mauve},
  breaklines=true,
  breakatwhitespace=true,
  tabsize=3
}
% Change language here
%\lstset{language=Java}

\title{nmPy Manual}
\author{Simon Schneider}
\date{September 2018}

\begin{document}

\maketitle
\tableofcontents
\section{Introduction}
What it is all about
\section{Run Synthetics}
This section describes how to use nmPy to calculate synthetic data using mdcpl,
 matdiag and synseis.
\begin{lstlisting}
from frospy.core.modes import read as read_modes
from frospy.SetupRuns import settings
from frospy.SetupRuns.setup import build_synthetics

args = settings.default_config()
args['intype'] = 'synthetics'
args['inmodel'] = 'S20RTS'
args['rundir'] = '//nfs/stig/simons/splitting/tests/synthetics/00s10-00t11/Z'
args['confdir'] = args['rundir']
args['modes'] = read_modes().select(name='0S10')
args['modes'] += read_modes().select(name='0T11')
args['cross_coupling'] = 1
args['cst_max'] = 12
args['cmt'] = ['060994A']
# Alternative datadir:
args['datadir'] = '//nfs/stig/simons/alldata/VHZ/'

setup = build_synthetics(args, remove_existing=True, run_code=True)
\end{lstlisting}

\section{Get Data}
\subsection{Download}
\subsection{Cleaning and Conversion}
\begin{lstlisting}
remtidah = '~/bin/remtidah'

from frospy.util.data_request import process_downloader_data
from frospy.util.read import read_std_cat

inv = None
cat = read_std_cat('011012A')
components = ["N", "E", "Z"]
sampling_rate = 0.1
localfolder = '/data/simons/3-comp-data/120hrs/VH_raw'
inspected_only = False
remove_deltas = True
rotate_traces = True
rm_response = True
rm_tidal = True
keep_longest_traces = True
cut_to_same_length = False

process_downloader_data(inv, cat, components, sampling_rate, localfolder,
                        inspected_only,
                        remove_deltas,
                        rotate_traces,
                        rm_response,
                        rm_tidal,
                        keep_longest_traces,
                        cut_to_same_length)
\end{lstlisting}

\subsection{Select Data Sets}
\begin{lstlisting}
from frospy.preprocessing.data_correction import select_station
st = select_station('file.ahx', tw=[5, 60], fw=[1, 4], min_meansnr=1.2)
\end{lstlisting}
Will loop over all noise windows defined in
\lstinline[columns=fixed]{nmpy/data/AD/noisewindows.dat}. Sets the window
between 2 noisewindows as signalwindow, calculates the maximum in this window.
Does the same with the noise in the 2 neighbouring windows, picks the biggest
peak in these. The SNR is defined as the ratio $\frac{max(Signal)}{min(Noise)}$
of those values. The code repeats that for all signal windows in the given
frequency interval \lstinline[columns=fixed]{fw=[frequency 1, frequency 2]}
and calulates the average of the SNR's. If it is smaller then
\lstinline[]{min_meansnr} the station will be deleted.

\paragraph{Each new event}

\begin{itemize}
\item calculate 0-2 mHz broadband
\item look at whole spectrum
\item delete bad stations permanently
\item look for globally evenly distributed station pattern
\item Check for events that occurred after the earthquake of interest (1.5-2 Mw difference)
\end{itemize}

\paragraph{Stations}
\begin{itemize}
\item remove (boxcar/delta pulse) glitches automatically
\item mark maximum tw per station and per event across all stations
\end{itemize}

\paragraph{Picking}
\begin{itemize}
\item loop over freq bands (0-1, 1-2, 3-4,… mHz)
\item define stations according to these bands
\item have a list of tw for each band
\item pick fw and tw (blindly)
\item Set tw$_{end}$ at first to qcycle length
\item later try to set it bigger, if it still makes sense
\item look at snr and m, which will depend on the mode type (mantle, IC, ...)
\item select and run inversion
\item Delete outlier stations (histogram of misfits), and run inversion again
\end{itemize}
It will be more important to have more events picked, than to have more segments of the same event
For this, spectrum can be used, by running it with e.g.:
\begin{lstlisting}
data = '060994A.ahx'
spectrum(data=data, fw=[0.3, 2], tw=[5, 60], minispec=True)
\end{lstlisting}

\paragraph{Picking for FSI}

Start with Q-cycle and extend time windows later
Choose between multiple modes in one segment or cutting to one mode per segment
If multiple modes in one segment: use multiple time windows in which the different modes are dominant
If one mode (maybe two) per segment: use Q-cycle at first.
Maybe downsample to 100 seconds for low frequency range (e.g. 0-3 mHz)

\section{Spectrum}
\subsection{Plotting}
\subsection{Pick Segments}

\section{CST/FSI Inversion}
\subsection{Build CST Inversion}
\begin{lstlisting}
from frospy.SetupRuns.damping_test import run_inversion
args = default_config()
allmodes = read_modes()
motherdir = '//nfs/stig/simons/splitting/tests'

run = 'ANTO'
args['comp'] = ['T', 'Z']
args['number_of_tasks'] = 10
args['cst_max'] = 4
args['cross_coupling'] = 1
args['autodatadir'] = '//nfs/stig/simons/splitting/tests/synthetics/00s10-00t11'

Modes = [['0T11', '0S10']]
damping = np.logspace(-4, 2, 7)[2:]
for modes in Modes:
    run_inversion(modes, args, damping, run, motherdir)
\end{lstlisting}

If you want to start the inversion from an existing file just change
\lstinline[columns=fixed]{args['modeldir']} to the path where the file is saved.
This file has to be the same format of course.

\subsection{Analyse Inversion Results}
\end{document}
